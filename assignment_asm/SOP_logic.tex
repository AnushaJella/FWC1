\documentclass{article}

% Language setting
% Replace `english' with e.g. `spanish' to change the document language
\usepackage[english]{babel}

% Set page size and margins
% Replace `letterpaper' with `a4paper' for UK/EU standard size
\usepackage[letterpaper,top=2cm,bottom=2cm,left=3cm,right=3cm,marginparwidth=1.75cm]{geometry}

% Useful packages
\usepackage{multicol}
\usepackage{amsmath}
\usepackage{graphicx}
\usepackage{array}
\usepackage{blindtext}
\usepackage{karnaugh-map}
\usepackage{circuitikz}
\usepackage[framemethod=tikz]{mdframed}
\usepackage[utf8]{inputenc}
\usepackage{tikz}
\usetikzlibrary{
  circuits.logic,
  circuits.logic.US,
  positioning
}

\usepackage[colorlinks=true, allcolors=blue]{hyperref}

\title{SOP expression for Boolean Function}
\author{Anusha Jella}

\begin{document}
\maketitle
\begin{multicols}{2}
\tableofcontents

\begin{abstract}
 This manual shows how to use Arduino with LED to represent Boolean function.
\end{abstract}
\section {Components}

%\begin{table}[]
    \centering
    \begin{tabular}{ |c |c |c |c |}
\hline
\textbf{Components} & \textbf{Value} & \textbf{Quantity} \\
\hline
 %Resistor & 220Ohm & 1 \\ 
 Arduino & UNO & 1 \\  
 %Seven segment Display &  & 1 \\
 %Decoder& 7447&1 \\
 LED & - & 1 \\
 Jumper wires&M-M &6\\
 Breadboard& &1\\
 \hline
 \end{tabular}
 \vspace{3mm}
 \centering
 \textbf{Table1.0}
    \label{table1}
%\end{table}

\section{Hardware}
\begin{flushleft}
\textbf{Problem 2.1} Plug the display to the breadboard and make the connections in Table \ref{table2}. Henceforth, all 5V and GND connections will be made from the breadboard. 
\end{flushleft}
\hspace{10mm}
\vspace{10mm}
 \centering
    \begin{tabular}{ |c |c |c |c |}
\hline
\textbf{Arduino} & \textbf{Breadboard} \\
\hline
 %Resistor & 220Ohm & 1 \\ 
 5V & Top Green \\  
 %Seven segment Display &  & 1 \\
 %Decoder& 7447&1 \\
 GND & Bottom Green\\
  \hline
 \end{tabular}
 \\
% \vspace{2mm}
\centering
 \textbf{Table 2}
    \label{table2}
    
\section{Software}
\begin{flushleft}
   \textbf{Problem 3.1.} Now make the connections as per
Table \ref{table3}.0 and execute the following program after
downloading.\\
\vspace{0.5cm}
%\framebox{
%\url{https://github.com/AnushaJella/assignment1}}
\begin{mdframed}
    \url{https://github.com/AnushaJella/assignment1/blob/main/codes/src/main.cpp}
\end{mdframed}
\vspace{2mm}
In the truth table in Table \ref{table4}.1, A,B,C are the
inputs and F is the output.Using boolean
logic
\\
\vspace{3mm}
\centering
    \begin{tabular}{ |c |c |c |c |}
\hline
\textbf{A} & \textbf{B}& \textbf{C}& \textbf{F(A,B,C)} \\
\hline
 %Resistor & 220Ohm & 1 \\ 
 0 & 0& 0 &1 \\ 
 0 & 0& 1 &0 \\
 0 & 1& 0 &0 \\
 0 & 1& 1 &1 \\ 
 1 & 0& 0 &1 \\ 
  1 & 0& 1 &0 \\ 
  1 & 1& 0 &0 \\ %Seven segment Display &  & 1 \\
 1 & 1& 1 &1 \\
  \hline
 \end{tabular}\\

 \vspace{3mm}
\centering
 \textbf{Table 3.1}
    \label{table4}
    \\
    
\begin{karnaugh-map}[4][2][1][$BC$][$A$]
        \minterms{0,4,3,7}
        \maxterms{1,2,5,6}
        %\indeterminants{2,5}
        \implicant{0}{4}
        \implicant{3}{7}
    \end{karnaugh-map}
    %\vspace{1mm}
    \centering
    \textbf{K-Map}
\begin{equation}
F=B'*C'+B*C
\label{eq1}
\end{equation}
\end{flushleft}

\vspace{0.5cm}
%\begin{table}[]
    \centering
    \begin{tabular}{ |c |c |c |c| }

%\textbf{Components} & \textbf{Value} & \textbf{Quantity} \\
\hline
 \textbf{} & A & B& C \\ 
 \hline
 \textbf{Input} & 0 & 0 & 0 \\
 \textbf{Arduino} & 6& 7 & 8 \\  

 \hline
 \end{tabular}\\
 \vspace{3mm}
\centering 
 \textbf{Table 3.0}
    \label{table3}
%\end{table}

\vspace{10mm}
 
    \vspace{3mm}
    \tikzstyle{branch}=[fill,shape=circle,minimum size=3pt,inner sep=0pt]
\title{logic circuit for F as in eq.\ref{eq1}}
\begin{tikzpicture}[label distance=2mm]

    %\node (x3) at (0,0) {$x_3$};
    \node (x2) at (1,0) {$A$};
    \node (x1) at (2,0) {$B$};
    \node (x0) at (3,0) {$C$};

    %\node[not gate US, draw, rotate=-90] at ($(x2)+(0.5,-1)$) (Not2) {};
    \node[not gate US, draw, rotate=-90] at ($(x1)+(0.5,-1)$) (Not1) {};
    \node[not gate US, draw, rotate=-90] at ($(x0)+(0.5,-1)$) (Not0) {};

    \node[and gate US, draw, logic gate inputs=nn,scale=0.05cm] at ($(x0)+(2,-2)$) (and1) {};
    \node[and gate US, draw, logic gate inputs=nn,scale=0.05cm] at ($(and1)+(0,-1)$) (and2) {};
   
    \node[or gate US, draw, logic gate inputs=nn, anchor=input 1,scale=0.05cm] at ($(and1.output -| and2.output)+(1,0)$) (or1) {};

\foreach \i in {1,0}
    {
        \path (x\i) -- coordinate (punt\i) (x\i |- Not\i.input);
        \draw (punt\i) node[branch] {} -| (Not\i.input);
    }
   
    \draw (x1) |- (and1.input 1);
    \draw (x1 |- and1.input 1) node[branch] {} -- (and1.input 1);
    \draw (Not1.output) |- (and2.input 1);
    \draw (Not1.output |- and2.input 1) node[branch] {} -- (and2.input 1);
    \draw (x0) |- (and1.input 2);
    \draw (Not0.output) |- (and2.input 2);
    \draw (Not0.output |- and2.input 2) node[branch] {} -- (and2.input 2);
    %\draw (and1.output) -- (or1.input 1);
    \draw (and1.output) -- ([xshift=0.5cm]and1.output) |- (or1.input 1);
    %\draw (and2.output) -- (or1.input 2);
    \draw (and2.output) -- ([xshift=0.5cm]and2.output) |- (or1.input 2);
   % \draw (Or1.output) -- (And2.input 1);
    %\draw (Nor1.output) -- ([xshift=0.5cm]Nor1.output) |- (And2.input 2);
    \draw (or1.output) -- ([xshift=0.5cm]or1.output) node[above] {$F$};

\end{tikzpicture}
\textbf{Problem 3.2.} Verify above code for all inputs from 000-111.
\end{multicols}{2}
\end{document}